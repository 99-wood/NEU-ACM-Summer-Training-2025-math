\documentclass[aspectratio=169]{beamer}  % 16:9 宽屏
\usefonttheme[onlymath]{serif} % 保留数学公式的手写体风格
\usepackage[UTF8]{ctex}                  % 中文支持
\usepackage{fontspec}                    % 字体设置
\usepackage{graphicx}                    % 插图支持
\usepackage{amsmath,amssymb,amsfonts,amsthm}            % 数学公式支持
\usepackage{tikz}                        % 绘图支持
\usepackage{hyperref}                    % 超链接支持

% 字体设置(可按需替换)
\setmainfont{Times New Roman}
\setsansfont{Arial}

% Beamer 主题(可改为其他主题如 Madrid、Frankfurt 等)
% \usetheme{Berlin}
\usetheme{Frankfurt}
% \usecolortheme{beetle	}

% 页脚设置
\setbeamertemplate{footline}[frame number]

% 封面信息
\title{数学基础}
\subtitle{2025 暑期留校集训}
\author{张佳栋}
\institute{东北大学\\计算机科学与工程学院}
\date{\today}
\logo{\includegraphics[width=1cm]{./figure/logo.png}}

\begin{document}

% 封面页
\begin{frame}
  \titlepage
\end{frame}

% 目录页
\begin{frame}{目录}
  \tableofcontents
\end{frame}


% 第一部分
\section{整除}
\begin{frame}{整除}
  \begin{itemize}
    \item 对于 $a \in \mathbb{Z}, a \neq 0$, 如果存在某个整数 $k$, 使得 $a = k \cdot d$, 则称 $d \mid a$ ($d$ 整除 $a$).
    \item $\mid$ 是整除符号.
    \item $d$ 是 $a$ 的约数, $a$ 是 $d$ 的倍数.
  \end{itemize}
\end{frame}

\begin{frame}{性质}
  \begin{itemize}
    \item 自反性: 对于任意整数 $n \neq 0$, 有 $n \mid n$.
    \item 反对称性: 若有 $a \mid b$ 且 $|a| \neq |b|$,则 $b \nmid a$.
    \item 传递性: 若有 $a \mid b, b \mid c$,则 $a \mid c$.
    \item $\forall a \in \mathbb{Z} \land a \neq 0, a \mid 0$.
    \item $\forall a \in \mathbb{Z}, 1 \mid a$.
    \item $a \mid b, a \mid c \Leftrightarrow
          \forall x, y \in \mathbb{Z}, a \mid (bx + cy)$.
    \item $m \neq 0, a \mid b \Leftrightarrow
          ma \mid mb$.
  \end{itemize}
\end{frame}

\begin{frame}
  
\end{frame}

\begin{frame}{研究意义}
  \begin{block}{核心贡献}
    \begin{itemize}
      \item 你做了什么
      \item 为什么重要
    \end{itemize}
  \end{block}
\end{frame}

% 第二部分
\section{方法}
\begin{frame}{方法框架}
  \begin{columns}
    \column{0.5\textwidth}
      \begin{itemize}
        \item 模型架构
        \item 算法流程
      \end{itemize}
    \column{0.5\textwidth}
      \includegraphics[width=\textwidth]{example-image}
  \end{columns}
\end{frame}

% 第三部分
\section{实验}
\begin{frame}{实验设计}
  \begin{itemize}
    \item 数据集
    \item 评估指标
    \item 对比方法
  \end{itemize}
\end{frame}

\begin{frame}{实验结果}
  \begin{figure}
    \includegraphics[width=0.8\textwidth]{example-image-a}
    \caption{实验对比图}
  \end{figure}
\end{frame}

% 结论
\section{总结}
\begin{frame}{总结与展望}
  \begin{itemize}
    \item 本文工作总结
    \item 不足与未来计划
  \end{itemize}
\end{frame}

% 致谢
\begin{frame}{致谢}
  \centering
  感谢您的聆听!\par
  \vspace{1cm}
  有问题欢迎交流!
\end{frame}

\end{document}
